\section{The advection equation}

In want to solve the following advection equation
\begin{equation}
  v_t + v v_x = 0,
  \label{eq_advection}
\end{equation}
where $v$ is velocity, $t$ is time, $x$ position, and $v_t$ is a short notation for the derivative $\frac{\partial v}{\partial t}$. We have the following initial condition
\[
  v(x, 0) = \cos x,
\]
and we want to solve the equation for values of $x$ and $t$ from the intervals
\begin{equation}
  -\pi < x < \pi, \quad \quad 0 \leq t \leq 1.4.
  \label{eq_x_t_ranges}
\end{equation}


\section{Using analytical solution}

Here we use analytical solution of \autoref{eq_advection}:
\begin{equation}
  v = \cos (x - v t ).
  \label{eq_analytical_solution}
\end{equation}
Our goal is to write a Fortran program that solves \autoref{eq_analytical_solution} for various values of $x$ and $t$ parameters from the intervals given by Inequalities \ref{eq_x_t_ranges}.

\subsection{Using Newton-Raphson method}

In order to solve \autoref{eq_analytical_solution} numerically, we use Newton-Raphson method:
\begin{equation}
  v_{n+1} = v_n - f(v_n, x, t) / f'(v_n, x, t),
  \label{eq_recurrence}
\end{equation}
where $n=1, 2, \dots, N_{max}$ is the iteration number with $N_{max}$ being the maximum number of iterations, and
\[
  f(v_n, x, t) = \cos (x - v_n t) - v_n,
\]
which is constructed by moving the terms of \autoref{eq_analytical_solution} to one side. Here $x$ and $t$ are fixed values that do not change during this calculation.

We begin the calculations by choosing a starting $v_1$ value and then use \autoref{eq_recurrence} to calculate $v_2$. Then we use $v_2$ to calculate $v_3$. This calculation is repeated until the absolute difference between two subsequent $v$ values is smaller than a chosen tolerance number $\epsilon$:
\[
  |{v_{n+1} - v_n}| < \epsilon.
\]
The calculations are also stopped and the program is terminated with an error if the number of iterations exceeds a chosen maximum number of iterations $N_{max}$. The program is also terminated if devision by zero or an overflow is detected as a result of calculating $v_{n+1}$ from \autoref{eq_recurrence}.

\subsection{Finding roots for different values of $x$ and $t$}

Our goal is to find roots of \autoref{eq_analytical_solution} for different values of $x$ and $t$ from the intervals defined by Inequalities \ref{eq_x_t_ranges}. In order to calculate these roots, we use Newton-Raphson method multiple times by choosing values of $x$ and $t$ from the following sequences:
\begin{align*}
  \{ x_{\textrm{start}}, x_{\textrm{start}} + \Delta x, x_{\textrm{start}} + 2 \Delta x, \dots, x_{\textrm{end}} \} \\
  \{ t_{\textrm{start}}, t_{\textrm{start}} + \Delta t, t_{\textrm{start}} + 2 \Delta t, \dots, t_{\textrm{end}} \},
\end{align*}
where $x_{\textrm{start}}$, $x_{\textrm{end}}$ are the smallest and largest $x$ values, and $t_{\textrm{start}}$, $t_{\textrm{end}}$ are the smallest and largest $t$ values that are supplied to the program by the user The values $\Delta x$, $\Delta t$ are the position and time steps that are calculated as follows:
\begin{align}
  \Delta x &= \frac{x_{\textrm{end}} - x_{\textrm{start}}}{n_x - 1} \label{eq_dx} \\
  \Delta t &= \frac{t_{\textrm{end}} - t_{\textrm{start}}}{n_t - 1}, \label{eq_dt}
\end{align}
where $n_x$ and $n_t$ are the number of position and time steps that are supplied to the program by the user.


\subsection{Writing the code}

Next, we write Fortran code to solve \autoref{eq_analytical_solution}. The code shown in \autoref{code_solve_v_equation} is a part of \code{find\_many\_roots} function. The code calculates the roots of \autoref{eq_analytical_solution} for various values of $x$ and $t$ parameters.

\noindent\begin{minipage}{\linewidth}
\begin{lstlisting}[caption={Solving $v = \cos (x - v t )$ equation for various values of parameters $x$ and $t$ (\code{root\_finder.f90}).},frame=tlrb,label={code_solve_v_equation}, numbers=left, firstnumber=160]
! Assign evenly spaced x and t values
call linspace(x_start, x_end, x_points)
call linspace(t_start, t_end, t_points)

! Calculate step sizes
dx = (x_end - x_start) / (nx - 1)
dt = (t_end - t_start) / (nt - 1)

! Calculate solutions for all values of x and t
do it = 1, nt
    do ix = 1, nx
        x = x_start + (ix - 1) * dx
        t = t_start + (it - 1) * dt

        root = find_root(options=options, x=x, t=t, success=success)

        if (.not. success) then
            ! Could not find root: return the problematic x and t
            error_x = x
            error_t = t
            return
        end if

        solution(ix, it) = root
    end do
end do
\end{lstlisting}
\end{minipage}

On \code{Line 161} we assign evenly spaced values between $x\_start$ and $x\_end$ for the $x$-coordinate and store then in the \code{x\_points} array. The values \code{x\_start} and \code{x\_end}. The number of points $nx$ are also supplied to the program by the user and they determine the size of the \code{x\_points} array.

On \code{Line 162} use the same technique to calculate the values of the $t$-coordinate and store them in $\code{t\_points}$ array.

On \code{Lines 165} and \code{166} we calculate the position and time steps $dx$ and $dt$ using Equations \ref{eq_dx} and \ref{eq_dt}.

Next, on \code{Lines 169} and \code{170} we use two loops to iterate over the range of indexes $it$ and $ix$. Inside the loops, on \code{Lines 171} and \code{172} we calculate the values of the $x$ and $t$ parameters for the current iteration.

On \code{Line 174} we call \code{find\_root} function. This function calculates a root of \autoref{eq_analytical_solution} using Newton-Raphson method and returns this root.

Newton-Raphson method does not guarantee to find a root. The case when the program does not find a root is handled on \code{Lines 176-181}. Here we store the problematic values of $x$ and $t$ and exit the subroutine with an error.

Alternatively, in case when the program does find a root, we store that value in a 2D array called \code{solution} on \code{Line 183}.

The end result of this part of the program is the $\code{solution}$ array filled with values of $v$ for all values of $x$ and $t$ parameters that the user has chosen. Alternatively, if the program could not find solution to \autoref{eq_analytical_solution} for just a single pair of $x$ and $t$, it exits with an error. In this case, the user is presented with an error message containing the values of $x$ and $t$ parameters for which Newton-Raphson method failed. The user can then re-run the program again with different values of starting $v_1$ value, tolerance and maximum number of iterations $N_{max}$ for Newton-Raphson method, until she finds settings for which Newton-Raphson is able to find solutions for all values of $x$ and $t$.
