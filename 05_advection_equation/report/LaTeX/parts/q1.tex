\section{The advection equation}

In want to solve the following advection equation
\begin{equation}
  v_t + v v_x = 0,
  \label{eq_advection}
\end{equation}
where $v$ is velocity, $t$ is time, $x$ position, and $v_t$ is a short notation for the derivative $\frac{\partial v}{\partial t}$. We have the following initial condition
\[
  v(x, 0) = \cos x,
\]
and we want to solve the equation for values of $x$ and $t$ from the intervals
\begin{equation}
  -\pi < x < \pi, \quad \quad 0 \leq t \leq 1.4.
  \label{eq_x_t_ranges}
\end{equation}


\section{Using analytical solution}

Here we use analytical solution of \autoref{eq_advection}:
\begin{equation}
  v = \cos (x - v t ).
  \label{eq_analytical_solution}
\end{equation}
Our goal is to write a Fortran program that solves \autoref{eq_analytical_solution} for various values of $x$ and $t$ parameters from the intervals given by Inequalities \ref{eq_x_t_ranges}.

\subsection{Using Newton-Raphson method}

In order to solve \autoref{eq_analytical_solution} numerically, we use Newton-Raphson method:
\begin{equation}
  v_{n+1} = v_n - f(v_n, x, t) / f'(v_n, x, t),
  \label{eq_recurrence}
\end{equation}
where $n=1, 2, \dots, N_{max}$ is the iteration number with $N_{max}$ being the maximum number of iterations, and
\[
  f(v_n, x, t) = \cos (x - v_n t) - v_n,
\]
which is constructed by moving the terms of \autoref{eq_analytical_solution} to one side. Here $x$ and $t$ are fixed values that do not change during this calculation.

We begin the calculations by choosing a starting $v_1$ value and then use \autoref{eq_recurrence} to calculate $v_2$. Then we use $v_2$ to calculate $v_3$. This calculation is repeated until the absolute difference between two subsequent $v$ values is smaller than a chosen tolerance number $\epsilon$:
\[
  |{v_{n+1} - v_n}| < \epsilon.
\]
The calculations are also stopped and the program is terminated with an error if the number of iterations exceeds a chosen maximum number of iterations $N_{max}$. The program is also terminated if devision by zero or an overflow is detected as a result of calculating $v_{n+1}$ from \autoref{eq_recurrence}.

\subsection{Finding roots for different values of $x$ and $t$}

Our goal is to find roots of \autoref{eq_analytical_solution} for different values of $x$ and $t$ from the intervals defined by Inequalities \ref{eq_x_t_ranges}. In order to calculate these roots, we use Newton-Raphson method multiple times by choosing values of $x$ and $t$ from the following sequences:
\begin{align*}
  \{ x_{\textrm{start}}, x_{\textrm{start}} + \Delta x, x_{\textrm{start}} + 2 \Delta x, \dots, x_{\textrm{end}} \} \\
  \{ t_{\textrm{start}}, t_{\textrm{start}} + \Delta t, t_{\textrm{start}} + 2 \Delta t, \dots, t_{\textrm{end}} \},
\end{align*}
where $x_{\textrm{start}}$, $x_{\textrm{end}}$ are the smallest and largest $x$ values, and $t_{\textrm{start}}$, $t_{\textrm{end}}$ are the smallest and largest $t$ values that are supplied to the program by the user The values $\Delta x$, $\Delta t$ are the position and time steps that are calculated as follows:
\begin{align*}
  \Delta x &= \frac{x_{\textrm{end}} - x_{\textrm{start}}}{n_x - 1} \\
  \Delta t &= \frac{t_{\textrm{end}} - t_{\textrm{start}}}{n_t - 1},
\end{align*}
where $n_x$ and $n_t$ are the number of position and time steps that are supplied to the program by the user.


% First, we move all terms of \autoref{eq_analytical_solution} to one side and call this an $f$ function:
% \[
%   f(x_n) = \cos (x - v t ) - v_
% \]
% We want to use a finite-difference (Euler) method to find an approximate solution of the heat equation
% \begin{equation}
%   T_t = k T_{xx}
%   \label{eq_heat_equation}
% \end{equation}
% with the initial condition
% \begin{equation}
%   T(x,0) = 100 \sin(\pi x / L),
%   \label{eq_initial_condition}
% \end{equation}
% and boundary conditions
% \begin{equation}
%   T(0,t) = T(L,t) = 0
%   \label{eq_boundary_conditions}
% \end{equation}
% for $L = 1 \ m$. Here we use a short notation for the derivatives: $T_t = \frac{\partial T}{\partial t}$.


% \subsection{The recurrence relation}

% Our goal is to find a recurrence relation that will allow us to calculate the value of temperature iteratively at each time index. In order to find this recurrence relation we use approximations for the derivatives. The time derivative approximation is
% \[
%   T_t(x, t) \approx \frac{T(x, t + \Delta t) - T(x, t)}{\Delta t},
% \]
% with the corresponding approximate expression
% \begin{equation}
%   (T_t)_j^n = \frac{T_j^{n + 1} - T_j^{n}}{\Delta t},
%   \label{eq_time_derivative_approximation}
% \end{equation}
% where the $j$ is the position index and $n$ is the time index:
% \begin{align*}
%   j &= 1, 2, \dots, n_x \\
%   n &= 1, 2, \dots, n_t,
% \end{align*}
% with $n_x$ and $n_t$ being the total number of position and time values respectively.
% Similarly, the approximate expression for the second position derivative is:
% \begin{equation}
%   (T_{xx})_j^n = \frac{T_{j+1}^n - 2 T_{j}^n + 2T_{j-1}^n}{\Delta x^2}.
%   \label{eq_space_derivative_approximation}
% \end{equation}
% Next, we substitute Equations \ref{eq_time_derivative_approximation} and \ref{eq_space_derivative_approximation} into \autoref{eq_heat_equation}:
% \[
%   \frac{T_j^{n + 1} - T_j^{n}}{\Delta t} = k \frac{T_{j+1}^n - 2 T_{j}^n + 2T_{j-1}^n}{\Delta x^2}.
% \]
% Finally, we solve for $T_j^{n + 1}$ and find the recurrence relation that we wanted
% \begin{equation}
%   \boxed{ T_j^{n + 1} = T_j^{n} + \alpha \big( T_{j+1}^n - 2 T_{j}^n + 2T_{j-1}^n \big), \quad \alpha = k \frac{\Delta t}{\Delta x^2}. }
%   \label{eq_recurrence_relation_heat_eq}
% \end{equation}


% \subsection{Writing the code}

% Next, we write Fortran code to solve the heat equation (\autoref{eq_heat_equation}). The code is shown in \autoref{code_solve_heat_equation}.


% \noindent\begin{minipage}{\linewidth}
% \begin{lstlisting}[caption={Solving a heat equation with forward-difference method (\code{heat\_equation.f90}).},frame=tlrb,label={code_solve_heat_equation}, numbers=left, firstnumber=77]
% ! Assign evenly spaced x values
% call linspace(x0, x1, x_points)

% ! Set initial conditions
% data(:, 1) = 100 * sin(pi * x_points / l)

% ! Set boundary conditions
% data(1, :) = 0
% data(nx, :) = 0

% ! Calculate numerical solution using forward differencing method
% do n = 1, nt - 1
%     data(2 : nx - 1, n + 1) = data(2 : nx - 1, n) &
%         + alpha * ( &
%             data(3 : nx, n) &
%             - 2 * data(2 : nx - 1, n) &
%             + data(1 : nx - 2, n) &
%         )
% end do
% \end{lstlisting}
% \end{minipage}

% On \code{Line 77} we assign evenly spaced values between $0$ and $1$ for the x-coordinate and store then in the \code{x\_points} array. The number of the values \code{nx} are supplied to the program by the user.

% Next, on \code{Line 81} use the initial condition from \autoref{eq_initial_condition}. Our temperature values are stored in a 2D array variable called \code{data}. Here we assign the temperatures to all the x-values corresponding to the first time index $n=1$.

% Similarly, on \code{Line 77} and \code{78} we use the boundary conditions (\autoref{eq_boundary_conditions}) by assigning zero temperatures to the ends of the rod. This is done by assigning zero for all time indexes in the \code{data} array corresponding to first (1) and last (\code{nx}) x-index.

% Finally, on \code{Lines 88-95} we iterate over time indexes (\code{n}) and use the recurrence relation from \autoref{eq_recurrence_relation_heat_eq} to assign the temperature value for the next time index (\code{n + 1}) using the temperatures that were calculated for the previous index of time (\code{n}).

% On \code{Lines 89-94} are using a so-called ``vectorized'' indexing syntax for the x-coordinate, such as \code{2:nx-1}. This allows the program to use SIMD processor instructions that perform multiple operations in one cycle. These instructions take advantage of the fact that x-values are located contiguously in memory (Fortran arrays are column-major, meaning the values from the first index of an array are stored one after another in memory). This index notation will make out calculation multiple times faster (this speed increase depends on type of SIMD instructions implemented in a specific processor) compared to an alternative implementation where we would iterate over the $x$ values using a loop.


% \subsection{Running the program and making plots}

% Instructions for compiling, running the program and plotting its results are located in the README.md file that comes with the source code.



% \subsection{Approximate solution and errors}

% We look at the approximate solution produced by our program by first plotting the solution for small number of x values ($nx = 5$) on \autoref{fix_solution_nx_5_alpha_0_1}.
% \begin{figure}[H]
%   \centering
%   \includegraphics[width=1.0\textwidth]{figures/{nx_5_alpha_0.10_solution}.pdf}
%   \caption{Approximate solution to the heat equation for $nx=5$ and $\alpha = 0.1$.}
%   \label{fix_solution_nx_5_alpha_0_1}
% \end{figure}
% We can see from \autoref{fix_solution_nx_5_alpha_0_1} that the the temperature of the rod decreases from a sinusoidal distribution with time, and the ends remain at $T=0$, as expected. 

% The corresponding errors are shown on \autoref{fix_errors_nx_5_alpha_0_1}. The errors were found by calculating the absolute value of the difference between the approximate solution and the exact solution, which is given by equation
% \[
%   T(x,t) = 100 e^{- \pi^2 k t / L^2} \sin(\pi x / L).
% \]
% We can see from \autoref{fix_errors_nx_5_alpha_0_1} that the errors are first increasing with time, reaching peak values of about $0.8 \ K$ at $t \approx 3000 \ s$, and then start to decrease. The data show that the errors are larger near the center of the rod.
% \begin{figure}[H]
%   \centering
%   \includegraphics[width=1.0\textwidth]{figures/{nx_5_alpha_0.10_errors}.pdf}
%   \caption{The absolute errors of the approximate solution of the heat equation for $nx=5$ and $\alpha = 0.1$.}
%   \label{fix_errors_nx_5_alpha_0_1}
% \end{figure}


% \subsubsection*{Solution for $21$ position values}

% Next, we use larger amount of position values, $nx = 21$ (we are counting both edges of the rod, so there are $20$ time steps) and $\alpha = 0.25$. The solution is shown on \autoref{fix_solution_nx_20_alpha_0_2_5}. We have calculated the solution for $300$ time values ($nt = 300$).
% \begin{figure}[H]
%   \centering
%   \includegraphics[width=1.0\textwidth]{figures/{nx_21_alpha_0.25_solution}.pdf}
%   \caption{Approximate solution to the heat equation for $nx=21$ and $\alpha = 0.25$.}
%   \label{fix_solution_nx_20_alpha_0_2_5}
% \end{figure}
% The general shape of the solution is not unlike the one we found for $nx=5$. We can see that the new $nx=21$ solution is smoother, which can be explained by smaller steps for position and time. The errors of the approximate solution for $nx=21$ are shown on \autoref{fix_errors_nx_20_alpha_0_2_5}.
% \begin{figure}[H]
%   \centering
%   \includegraphics[width=1.0\textwidth]{figures/{nx_21_alpha_0.25_errors}.pdf}
%   \caption{The errors of the approximate solution to the heat equation for $nx=21$ and $\alpha = 0.25$.}
%   \label{fix_errors_nx_20_alpha_0_2_5}
% \end{figure}
% We can see similar distribution of errors as before. This time, however, the peak values of errors are about $20$ times smaller, reaching maximum values of about $0.04 \ K$.


% \subsection{Solution for $\alpha > 0.5$}

% Here we want to see how values of $\alpha$ parameter larger than $0.5$ affect the solution. Specifically, we use $\alpha=0.562$, while keeping the same time step $\Delta t=27.4 \ s$ that we used for $nx = 21$. In order to keep $\Delta t$ unchanged we have to increase the number of position values from $nx=21$ to $nx=31$. Consequently, position step size decreases from $\Delta x = 0.05 \ m$ to $\Delta x \approx 0.033 \ m$. The solution is shown on \autoref{fix_solution_nx_20_alpha_0_6_5}.
% \begin{figure}[H]
%   \centering
%   \includegraphics[width=1.0\textwidth]{figures/{nx_31_alpha_0.56_solution}.pdf}
%   \caption{Approximate solution to the heat equation for $nx=31$ and $\alpha = 0.562$, showing signs of numerical instability from $t \approx 4800 \ s$.}
%   \label{fix_solution_nx_20_alpha_0_6_5}
% \end{figure}
% We can see form \autoref{fix_solution_nx_20_alpha_0_6_5} that initially, the program produces results similar to the previous case with $\alpha=0.25$. However, after about $t \approx 4800 \ s$, the $\alpha=0.562$ solution shows large fluctuations in temperature. These result are unrealistic, since the rood cools down with time, and it can not heat up in random locations without an external supply of energy. Errors of the $\alpha=0.562$ solution are shown on \autoref{fix_errrors_nx_31}. We can see that errors grow up to about $70 \ K$ at $t \approx 4800 \ s$, which confirms that there are issues with our solution.
% \begin{figure}[H]
%   \centering
%   \includegraphics[width=1.0\textwidth]{figures/{nx_31_alpha_0.56_errors}.pdf}
%   \caption{Errors of solution to the heat equation for $nx=31$ and $\alpha = 0.562$.}
%   \label{fix_errrors_nx_31}
% \end{figure}

% These problems in our numerical solution are likely to come from unbounded growth of rounding errors. A more detailed stability analysis using Von Neumann method is needed to show that the forward-difference method of solving \autoref{eq_heat_equation} is unstable for $\alpha > 0.5$. This condition limits our choice of the size of the position steps: we can not make $\Delta x$ as small as we like for same values of $\Delta t$. Fortunately, there are alternative methods of solving \autoref{eq_heat_equation}, such as backward-difference and Crank-Nicolson methods, that are always stable for any choice of $\Delta x$ and $\Delta t$.



% \subsection{Conclusion}

% We used forward-difference (Euler) method and solved a heat equation with initial and boundary conditions. We have found that smaller position and time steps result in smaller errors. However, we have also found that making our position step too small results in unlimited growth of errors.