\part{Overview}
\fancyhead[LE,RO]{Lab 2 report}

This is a Fortran program that estimates a single root of equation
\begin{equation}
  \cos(x) - x = 0
  \label{eq_root}
\end{equation}
using Newton-Raphson method. The method gives recurrence relation
\begin{equation}
  x_{x+1} = x_n - f(x_n) / f'(x_n),
  \label{eq_recurrence}
\end{equation}
where
\[
  f(x_n) = \cos(x_n) - x_n,
\]
and $x_n$ are the x values, and $n = 0, 1, 2, \dots, N_{max}$ is the iteration number with $N_{max}$ being the maximum number of iterations.

We begin the calculations by picking a starting $x_0$ and then use \autoref{eq_recurrence} to calculate $x_1$. Then we use $x_1$ to calculate $x_2$. This calculation is repeated until the absolute difference between two subsequent $x$ values is smaller than a chosen tolerance number $\epsilon$:
\[
  |{x_{n+1} - x_n}| < \epsilon.
\]

The calculations are stopped and the program is terminated with an error if the number of iterations exceeds a chosen maximum number of iterations. The program is also terminated if devision by zero or an overflow is detected as a result of calculating $x_{x+1}$ from \autoref{eq_recurrence}.

Instructions for compiling and running the program are located in the README.md file.


\section{Root finding function}

We implemented a function \code{approximate\_root} with interface shown in \autoref{code_approximate_root}.

\noindent\begin{minipage}{\linewidth}
\begin{lstlisting}[caption={Definition of a function for approximating a root of equation that is passed as input parameter (\code{newton\_raphson.f90}).},frame=tlrb,label={code_approximate_root}]
function approximate_root(x_start, func, derivative, tolerance, &
                          max_iterations, success) result(result)

...
end function
\end{lstlisting}
\end{minipage}

When calling \code{approximate\_root} function, we supply a function that calculates
\[
  f(x) = \cos(x) - x,
\]
as well as its derivative. This implementation allows to make \code{approximate\_root} function general and reuse it for calculating roots of other functions.


\section{Choosing initial x value}

We can estimate general location of the root of \autoref{eq_root} by evaluating $f(x) = \cos(x) - 1$ until we find two x values $x_a$ and $x_b$ for which $f$ has values of opposite signs. Intermediate value theorem guarantees that $f(x)=0$ for some $x \in [x_a, x_b]$, since $f$ is continuous. For example, we can chose our initial x value to be between $0$ and $\pi$, since $f(0)=1$ and $f(\pi/2) = -\pi/2$ have opposite signs.

