\section{Writing Fortran code}

We write a Fortran program that solves Burgers' equation. Instructions for compiling and running the program can be found in the README.md file that comes with the source code.


\section{Plotting solutions for sine initial conditions}

We calculate solutions to Burgers' equation calculated using Godunov's and Kurganov-Tadmor's methods. We use one hundred $x$ points, Courant factor of $0.5$ and sine initial conditions. The plots of the solutions are shown on Figures \ref{fig_sine_t_0},  \ref{fig_sine_t_0.5} and \ref{fig_sine_t_1.0} for time values of $t=0, 0.5, 1 \ s$.

A movie of the solution is available here:

\texttt{\footnotesize https://youtu.be/Pp4WZkqskdg}

\begin{figure}[H]
    \centering
    \includegraphics[width=1\textwidth]{figures/{01_sine_time_0.0}.pdf}
    \caption{Solutions of Burgers' equation with sine initial conditions at $t = 0 \ s$.}
    \label{fig_sine_t_0}
\end{figure}

\begin{figure}[H]
    \centering
    \includegraphics[width=1\textwidth]{figures/{02_sine_time_0.5}.pdf}
    \caption{Solutions of Burgers' equation with sine initial conditions at $t = 0.5 \ s$.}
    \label{fig_sine_t_0.5}
\end{figure}

\begin{figure}[H]
    \centering
    \includegraphics[width=1\textwidth]{figures/{03_sine_time_1.0}.pdf}
    \caption{Solutions of Burgers' equation with sine initial conditions at $t = 1.0 \ s$.}
    \label{fig_sine_t_1.0}
\end{figure}


\section{Bonus: plotting solutions for square initial conditions}

Next, we solve Burgers' equation, starting with square initial conditions, using the same parameters as before. The plots of the solutions are shown on Figures \ref{fig_square_t_0},  \ref{fig_square_t_0.5} and \ref{fig_square_t_1.0}

A movie of the solution is available here:

\texttt{\footnotesize https://youtu.be/oekOnANB0D0}

\begin{figure}[H]
    \centering
    \includegraphics[width=1\textwidth]{figures/{04_square_time_0.0}.pdf}
    \caption{Solutions of Burgers' equation with square initial conditions at $t = 0 \ s$.}
    \label{fig_square_t_0}
\end{figure}

\begin{figure}[H]
    \centering
    \includegraphics[width=1\textwidth]{figures/{05_square_time_0.5}.pdf}
    \caption{Solutions of Burgers' equation with square initial conditions at $t = 0.5 \ s$.}
    \label{fig_square_t_0.5}
\end{figure}

\begin{figure}[H]
    \centering
    \includegraphics[width=1\textwidth]{figures/{06_square_time_1.0}.pdf}
    \caption{Solutions of Burgers' equation with square initial conditions at $t = 1.0 \ s$.}
    \label{fig_square_t_1.0}
\end{figure}


\section{Analyzing results}

It can be seen from Figures \ref{fig_sine_t_1.0} and \ref{fig_square_t_1.0} that Burgers' and Kurganov-Tadmor's solution appear to agree for most values of the domain. However, the two methods are slightly different in shock regions -- at values of $x$ where velocity is changing rapidly. In these regions Kurganov-Tadmor's solution is more smooth than Godunov's.
