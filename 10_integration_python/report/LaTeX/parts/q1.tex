\section{Writing Fortran code}

We write a Python program that solves Lane-Emden equation. Instructions for running the unit tests and making plots can be found in the README.md file that comes with the source code.

For simplicity we wrote our program using functions and not classes. If classes are required for assessment of these work, the code is located in \code{classes.zip} file.


\section{Plots}

\subsection*{Task 1}

On \autoref{fig_task_1} we plot solution to Lane-Emden equation for scaled density, calculated using Euler method.

\begin{figure}[H]
    \centering
    \includegraphics[width=1\textwidth]{figures/{01_lane_emden}.pdf}
    \vspace*{-10mm}
    \caption{Solution to Lane-Emden equation.}
    \label{fig_task_1}
\end{figure}
\pagebreak


\subsection*{Task 2: changing step size}

On Fig. \ref{fig_task_2a}-\ref{fig_task_2c} we plot solutions to Lane-Emden equation for scaled density and its derivative. The solutions were calculated using different step sizes.

\begin{figure}[H]
    \centering
    \includegraphics[width=1\textwidth]{figures/{02a_density_vs_radius_h_0.1}.pdf}
    \vspace*{-10mm}
    \caption{Solutions to Lane-Emden equation for step size $h=0.1$.}
    \label{fig_task_2a}
\end{figure}

\begin{figure}[H]
    \centering
    \includegraphics[width=1\textwidth]{figures/{02b_density_vs_radius_h_0.01}.pdf}
    \vspace*{-10mm}
    \caption{Solutions to Lane-Emden equation for step size $h=0.01$.}
    \label{fig_task_2b}
\end{figure}

\begin{figure}[H]
    \centering
    \includegraphics[width=1\textwidth]{figures/{02c_density_vs_radius_h_0.001}.pdf}
    \vspace*{-10mm}
    \caption{Solutions to Lane-Emden equation for step size $h=0.001$.}
    \label{fig_task_2c}
\end{figure}
\pagebreak

\subsection*{Task 3: approximate and exact solutions}

On Fig. \ref{fig_task_3a}-\ref{fig_task_3c} we plot approximate and exact solutions to Lane-Emden for values of polytropic index $n=0, 1, 5$.

\begin{figure}[H]
    \centering
    \includegraphics[width=1\textwidth]{figures/{03a_density_vs_radius_n_0}.pdf}
    \vspace*{-10mm}
    \caption{Solutions to Lane-Emden equation for polytropic index $n=0$.}
    \label{fig_task_3a}
\end{figure}

\begin{figure}[H]
    \centering
    \includegraphics[width=1\textwidth]{figures/{03b_density_vs_radius_n_1}.pdf}
    \vspace*{-10mm}
    \caption{Solutions to Lane-Emden equation for polytropic index $n=1$.}
    \label{fig_task_3b}
\end{figure}

\begin{figure}[H]
    \centering
    \includegraphics[width=1\textwidth]{figures/{03c_density_vs_radius_n_5}.pdf}
    \vspace*{-10mm}
    \caption{Solutions to Lane-Emden equation for polytropic index $n=5$.}
    \label{fig_task_3c}
\end{figure}
\pagebreak

\subsection*{Task 6: comparing integrators}

Next, we calculate radius and density derivative at the surface of stellar model using Euler, Improved Euler and Runge-Kutta integrators, as well as exact solutions to Lane-Emden equation. Results are shown in \autoref{q6_comparing_integrators}.


\begin{table}[!ht]
\centering
\caption{Radius and density derivatives at the surface of a stellar model calculated with Euler, Improved Euler, Runge-Kutta integrators and exact solutions to Lane-Emden equation.}
\begin{tabular}{|c|c|c|c|}
\hline
\textbf{Step size} & \textbf{Method}  & \textbf{Radius at surface, $\xi_1$}   & \textbf{Density derivative at surface $\frac{d \theta}{d \xi} \big|_{\xi_1}$} \\
\hline
0.1       & Euler          & 3.10000 &  -0.338832 \\
\hline
0.1       & Improved Euler & 3.10000 &  -0.325917 \\
\hline
0.1       & Runge-Kutta    & 3.10000 &  -0.326620 \\
\hline
0.1       & Exact          & 3.14159 &  -0.318309 \\
\hline
0.01      & Euler          & 3.12999 &  -0.321732 \\
\hline
0.01      & Improved Euler & 3.13999  & -0.318625 \\
\hline
0.01      & Runge-Kutta    & 3.13999  & -0.318632 \\
\hline
0.01      & Exact          & 3.14159 &  -0.318309 \\
\hline
0.001     & Euler          & 3.14099  & -0.318535 \\
\hline
0.001     & Improved Euler & 3.14099  & -0.318429 \\
\hline
0.001     & Runge-Kutta    & 3.14099  & -0.318429 \\
\hline
0.001     & Exact          & 3.14159 &  -0.318309 \\
\hline
\end{tabular}
\label{q6_comparing_integrators}
\end{table}

\pagebreak


\subsection*{Task 7: stellar model with physical units}

Finally, we calculate density, temperature and pressure for a two-solar-mass stellar model with polytropic index $n=3$, central density $\rho_c = \SI{1e5}{\kg \ \m^{-3}}$ and mean molecular weight $\mu = 1.4$. Results are shown on Fig. \ref{fig_task_7a}-\ref{fig_task_7c}.

In our calculations and plots we are using SI and not CGS units because SI units are recommended by International Astronomical Union:

\footnotesize{ \url{https://www.iau.org/publications/proceedings_rules/units/} }


\begin{figure}[H]
    \centering
    \includegraphics[width=1\textwidth]{figures/{07a_density}.pdf}
    \vspace*{-10mm}
    \caption{Density of a stellar model calculated using Lane-Emden equation.}
    \label{fig_task_7a}
\end{figure}

\begin{figure}[H]
    \centering
    \includegraphics[width=1\textwidth]{figures/{07b_temperature}.pdf}
    \vspace*{-10mm}
    \caption{Temperature of a stellar model calculated using Lane-Emden equation.}
    \label{fig_task_7b}
\end{figure}

\begin{figure}[H]
    \centering
    \includegraphics[width=1\textwidth]{figures/{07c_pressure}.pdf}
    \vspace*{-10mm}
    \caption{Pressure of a stellar model calculated using Lane-Emden equation.}
    \label{fig_task_7c}
\end{figure}
\pagebreak
